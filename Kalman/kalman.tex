\documentclass[12pt, a4paper, oneside]{ctexbook}
\usepackage{amsmath, amsthm, amssymb, bm, graphicx, hyperref, mathrsfs}

\title{{\Huge{\textbf{卡尔曼滤波}}}\\——基础卡尔曼滤波原理推导}
\author{杜轩}
\date{\today}
\linespread{1.5}
\newtheorem{theorem}{定理}[section]
\newtheorem{definition}[theorem]{定义}
\newtheorem{lemma}[theorem]{引理}
\newtheorem{corollary}[theorem]{推论}
\newtheorem{example}[theorem]{例}
\newtheorem{proposition}[theorem]{命题}

\begin{document}

\maketitle

\pagenumbering{roman}
\setcounter{page}{1}

\begin{center}
    \Huge\textbf{前言}
\end{center}~\

卡尔曼滤波是通过预测和测量得到最优的结果,其算法分为预测和矫正两个部分

\textbf{预测:}
\begin{align*}
    \text{先验:}\hat{x_k}^-&=A\hat{x_{k-1}}+Bu_{k-1}\\
    \text{先验误差协方差:}P_k^-&=AP_k^-A^T+Q
\end{align*}

\textbf{校正:}
\begin{align*}
    \text{卡尔曼增益:}K_k&=\frac{P_k^-H^T}{HP_k^-H^T+R}\\
    \text{后验估计:}\hat{x_k}&=\hat{x_k}^-+K_k(Z_k-H\hat{x_k}^-)\\
    \text{更新误差协方差:}P_k&=(I-K_kH)P_k^-
\end{align*}
~\\
\begin{flushright}
    \begin{tabular}{c}
        杜轩\\
        \today
    \end{tabular}
\end{flushright}

\newpage
\pagenumbering{Roman}
\setcounter{page}{1}
\tableofcontents
\newpage
\setcounter{page}{1}
\pagenumbering{arabic}

\chapter{基础卡尔曼滤波原理推导}
\section{模型}
    对于模型(状态转移方程)
    \begin{align*}
        x_k&=Ax_{k-1}+Bu_{k-1}+w_{k-1}\\
        z_k&=Hx_k+v_k
    \end{align*}
    其中$u$为输入,$z$是输出,A,B,H已知,$w$和$v$是干扰满足如下高斯分布
    \begin{align*}
        w\sim (0,Q)\\
        v\sim (0,R)
    \end{align*}
我们用$\hat{x_k}$表示预测值,其中$\hat{x_k}^-$代表先验估计(通过上一步后验估计和状态转移方程计算),$\hat{x_k}$代表后验估计(通过先验估计和测量得到)\\
在忽略$w$的噪声干扰情况下我们可以得到先验估计(就是通过已知的状态空间方程计算)则满足$\hat{x_k}^-=A\hat{x_{k-1}}+Bu_{k-1}$\\
因为$x_k$测量不到所以通过$x_k=H^{-1}z_k$得到忽略测量干扰

令$\hat{x_k} =\hat{x_k}^-+R(H^{-1}z_k-\hat{x_k}^-)$,(R属于0到1),这样R接近零则表示预测值更信任预测,反之解决1更信任测量。
令$K_k=RH$则$\hat{x_k} =\hat{x_k}^-+K_k(z_k-H\hat{x_k}^-)$,其中$k_k$为卡尔曼增益
\section{推导}
我们定义误差(真实值和预测值得误差)为$e_k=\hat{x_k}-x_k$,其中$x_k$代表真实值。
我们计算误差为
\begin{align*}
    e_k&=\hat{x_k}-x_k\\
        &=\hat{x_k}^-+K_k(z_k-H\hat{x_k}^-)-x_k\\
        &=(I-K_kH)\hat{x_k}^-+K_kz_k-x_k\\
        &=(I-K_kH)\hat{x_k}^-+K_k(Hx_k+v_k)-x_k\\
        &=(I-K_kH)\hat{x_k}^--(I-K_kH)x_k+K_kv_k\\
        &=(I-K_kH)(\hat{x_k}^--x_k)+K_kv_k\\
\end{align*}
我们令$e_k^-=\hat{x_k}^--x_k$为先验误差

计算误差期望
\begin{align*}
    E(e_k)&=E((I-K_kH)(\hat{x_k}^--x_k))+K_kE(v_k)\\
          &=E((I-K_kH)(A\hat{x_{k-1}}+Bu_{k-1}-Ax_{k-1}-Bu_{k-1}-w_{k-1}))\\
          &=(I-K_kH)AE(\hat{x_{k-1}}-x_{k-1})-E(w_{k-1})\\
\end{align*}
我们假设初始误差期望为0,即可得$E(e_k)=0$

我们要使得结果最优即方差最小估计我们计算$e_k$得协方差矩阵$P_k$,其对角线即为方差。
\begin{align*}
    P_k&=E(ee^T)\\
     &=E(((I-K_kH)(\hat{x_k}^--x_k)+K_kv_k)((I-K_kH)(\hat{x_k}^--x_k)+K_kv_k)^T)\\
     &=E((I-K_kH)e_k^-(e_k^-)^T(I-K_kH)^T+(I-K_kH)e_k^-(v_k)^TK_k^T+K_kv_k(e_k^-)^T(I-K_kH)^T\\
     &+K_kv_k(v_k)^TK_k^T)\\
\end{align*}
因为$e_k^-=\hat{x_k}^--x_k=A\hat{x_{k-1}}+Bu_{k-1}-Ax_{k-1}-Bu_{k-1}-w_{k-1}$,因为$v_k$影响k时刻得测量,且$V_k$相互独立,则$e_k$和$v_k$无关从而$E(e_kv_k)=E(e_k)E(v_k)$
则
\begin{align*}
    P_k&=E((I-K_kH)e_k^-(e_k^-)^T(I-K_kH)^T+(I-K_kH)e_k^-(v_k)^TK_k^T+K_kv_k(e_k^-)^T(I-K_kH)^T\\
       &+K_kv_k(v_k)^TK_k^T)\\
     &=E((I-K_kH)e_k^-(e_k^-)^T(I-K_kH)^T)+E((I-K_kH)e_k^-(v_k)^TK_k^T)+E(K_kv_k(e_k^-)^T(I-K_kH)^T)\\
     &+E(K_kv_k(v_k)^TK_k^T)\\
     &=E((I-K_kH)e_k^-(e_k^-)^T(I-K_kH)^T)+K_kRK_k^T
\end{align*}
我们令$P_k^-=e_k^-(e_k^-)^T$称为先验协方差矩阵,则可得
\begin{align*}
    P&=(I-K_kH)P_k^-(I-K_kH)^T+K_kRK_k^T\\
     &=P_k^--K_kHP_k^--P_k^-H^TK_k^T+K_kHP_k^-H^TK_k^T+K_kRK_k^T\\
\end{align*}
我们需要方差最小,即$tr(P)$(代表矩阵得迹)最小
\begin{align*}
    tr(P)&=tr(P_k^-)-tr(K_kHP_k^-)-tr(P_k^-H^TK_k^T)+tr(K_kHP_k^-H^TK_k^T)+tr(K_kRK_k^T)\\
\end{align*}
矩阵的转置迹不变则
\begin{align*}
    tr(P)=tr(P_k^-)-2tr(K_kHP_k^-)+tr(K_kHP_k^-H^TK_k^T)+tr(K_kRK_k^T)
\end{align*}
我们进行求偏导通过
\begin{align*}
\frac{d tr(P_k)}{d K_k} =0
\end{align*}
求解出$K_k$的值,(由于矩阵求导公式$\frac{d AB}{d A}=B^T$和$\frac{d ABA^T}{d A}=2AB$)
\begin{align*}
    \frac{d tr(P_k)}{d K_k} =0-2tr(HP_k^-)^T+2tr(K_kHP_k^-H^T)+tr(2K_kR)=0
\end{align*}
于是得到如下公式
\begin{align*}
    K_kR+K_kHP_k^-H^T&=HP_k^-\\
    K_k&=\frac{HP_k^-}{R+HP_k^-H^T}
\end{align*}
下一个问题是怎么计算先验误差
\begin{align*}
P_k^-&=E(e_K^-(e_k^-)^T)\\
     &=E((A\hat{x_{k-1}}+Bu_{k-1}-Ax_{k-1}-Bu_{k-1}-w_{k-1})(A\hat{x_{k-1}}+Bu_{k-1}-Ax_{k-1}-Bu_{k-1}-w_{k-1})^T)\\
     &=E((Ae_{k-1}-w_{k-1})(Ae_{k-1}-w_{k-1})^T)\\
     &=E(Ae_{k-1}e_{k-1}^TA^T-w_{k-1}e_{k-1}^TA^T-Ae_{k-1}w_{k-1}^T+w_{k-1}w_{k-1}^T)\\
     &=AP_{k-1}A^T+Q
\end{align*}
其中$w_{k-1}$和$e_{k-1}$是独立的(k时刻的干扰影响k+1时刻的误差)则
\begin{equation*}
P_k^-=AP_{k-1}A^T+Q
\end{equation*}
因此
\textbf{预测:}
\begin{align*}
    \text{先验:}\hat{x_k}^-&=A\hat{x_{k-1}}+Bu_{k-1}\\
    \text{先验误差协方差:}P_k^-&=AP_k^-A^T+Q
\end{align*}
因为
\begin{align*}
    P&=P_k^--K_kHP_k^--P_k^-H^TK_k^T+K_kHP_k^-H^TK_k^T+K_kRK_k^T\\
     &=P_k^--K_kHP_k^--P_k^-H^TK_k^T+K_k(HP_k^-H^T+R)K_k^T\\
     &=P_k^--K_kHP_k^--P_k^-H^TK_k^T+P_k^-H^TK_k^T\\
     &=P_k^--K_kHP_k^-\\
\end{align*}
\textbf{校正:}
\begin{align*}
    \text{卡尔曼增益:}K_k&=\frac{P_k^-H^T}{HP_k^-H^T+R}\\
    \text{后验估计:}\hat{x_k}&=\hat{x_k}^-+K_k(Z_k-H\hat{x_k}^-)\\
    \text{更新误差协方差:}P_k&=(I-K_kH)P_k^-
\end{align*}
\end{document}